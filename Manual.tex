\documentclass[twoside]{urjc-tfg}


\newcommand{\grado}{Grados en Ciber, IA, II, IC, etc}
\newcommand{\titulotrabajo}{Manual de uso\\ Plantilla para\\ TFG/ TFM}
\newcommand{\tipotrabajo}{}
\newcommand{\curso}{Curso 20XX-20XX}
\newcommand{\autor}{Joaquín Arias}
\newcommand{\tutor}{}
\newcommand{\cotutor}{} %% Completar si procede

\bibliography{bibliografia}

\sloppy

%%%%%%%%%%%%%%%%%%%%%%%%%%%%%%% Inicio del documento
\begin{document}

\maketitle


%%%%%%%%%%%%%%%%%%%%%%%%% FrontMatter
\cleardoublepage
\pagenumbering{roman}
\setcounter{page}{1}

\chapter*{Resumen}
\addcontentsline{toc}{chapter}{Resumen}


Este documento tiene como objetivo servir de plantilla y manual de uso
para la redacción de trabajos fin de Grado (TFG) y/o Máster (TFM) de
la Universidad Rey Juan Carlos de Madrid.

El manual está compuesta por cuatro archivos:
\begin{itemize}
\item \url{Manual.tex}: contiene el código \LaTeX que genera este
  documento y que tiene como finalidad ayudar al alumno a resolver
  algunas dudas iniciales en el uso de \LaTeX.

\item \url{urjc-tfg.cls}: plantilla que define la clase a importar por
  en el documento del TFG/TFM.

\item \url{urjc-logo.jpg}: archivo con el logotipo de la
  Universidad Rey Juan Carlos. Se usa en la portada de la plantilla y
  como ejemplo para explicar al inserción de imágenes en el
  Capítulo~\ref{sec:image}.

\item \url{bibliografia.bib}:\footnote{Por conveniencia el código de
    \code|bibliografia.bib| está disponible en Apéndice~\ref{app:bib}}
  archivo con las referencias bibliográficas que se usan en este
  manual. Están codificadas usando BibTeX, su uso se explica en el
  Capítulo~\ref{sec:bib}.
\end{itemize}



% \chapter*{Agradecimientos}
% \addcontentsline{toc}{chapter}{Agradecimientos}

% Breves agradecimientos o dedicatoria.



%%%%%%%%%%%%%%%%%%%%%%%%% Índices
{
  \setlength{\parskip}{1pt}

  % Índice de contenidos
  \cleardoublepage
  \phantomsection
  \addcontentsline{toc}{chapter}{\contentsname}
  \tableofcontents

  % Índice de tablas (OPCIONAL)
  \cleardoublepage
  \phantomsection
  \addcontentsline{toc}{chapter}{\noindent \listtablename}
  \listoftables

  % Índice de figuras (OPCIONAL)
  \cleardoublepage
  \phantomsection
  \addcontentsline{toc}{chapter}{\listfigurename}
  \listoffigures
}


%%%%%%%%%%%%%%%%%%%%%%%%%%%%%% MainMatter
\cleardoublepage
\pagenumbering{arabic}
\setcounter{page}{1}

La creación de documentos con \LaTeX requiere el conocimiento de
conceptos básicos de programación e implica, como veremos en el
Capítulo~\ref{cap:tools}, la utilización de un conjunto de
herramientas, para por ejemplo editar textos, visualizar archivos
\code{pdf} y/o la gestión de referencias bibliográficas.

\LaTeX es un lenguaje de programación que permite la composición de
textos, es software libre bajo licencia
LPPL.\footnote{\url{http://www.latex-project.org/lppl/}.} Fue
desarrollado por Leslie Lamport en 1984 y consiste en un conjunto de
macros de \TeX, un lenguaje de composición tipográfica creado por
Donald Knuth (\LaTeX y \TeX se codifican resp. con las palabras claves
\code{\LaTeX} y \code{\TeX}). Una de las principales ventajas de este
lenguaje de programación es su versatilidad a la hora de representar
expresiones matemáticas.\footnote{Para más información visitar
  \url{https://manualdelatex.com/tutoriales/ecuaciones}.}

En relación al objetivo principal de esta plantilla, la realización de
la memoria de un Trabajo Fin de Grado y/o Máster en la Universidad Rey
Juan Carlos, recomiendo consultar la normativa vigente en cada momento
y prestando especial atención a las características especiales del
grado y/o máster concreto:

\begin{itemize}
\item Para TFG consultar
  \url{https://bit.ly/3fZf4Lb}.
\item Para TFM consultar
  \url{https://bit.ly/3i2fTpn}.
\end{itemize}

Es importante recordar que este manual pretende servir de guía para la
redacción de una memoria en \LaTeX utilizando la plantilla como ayuda,
para profundizar en los distintos aspectos que trataremos en este
manual existen múltiples recursos como por ejemplo el
\emph{Comprehensive TEX Archive Network} (CTAN) accesible a través de
\url{https://ctan.org/starter} y que cuenta con varios manuales
(alguno en español) y con multitud de paquetes para mejorar la calidad
de los documentos generados con \LaTeX.

\section{Estructura del Manual}
\label{sec:man}

Explicar que hay en cada capítulo...


\chapter{Herramientas para compilar un documento escrito en \LaTeX}
\label{cap:tools}

\section{Plataforma online: Overleaf}

\section{IDE para Windows/Mac/Linux: TeXstudio}

\section{Editor de texto plano (p.ej., Emacs) + AUC \TeX}



\chapter{Pequeño Manual de \LaTeX}

\section{Organización}

\subsection{Capítulos, Secciones y otros}

\subsection{Referencias a elementos dentro del documento}

\section{Formato de texto}

\subsection{Negrita, Cursiva y Enlaces}

\subsection{Tamaño de Letra y Control Espacios}

\subsection{Listas y enumeraciones}

\section{Incluir elementos visuales}

\subsection{Imágenes}
\label{sec:image}

\subsection{Código}

\subsubsection{Código en Python}

\begin{lstlisting}[style=Python]
# A continuación presentamos una implementación de factorial
# que guarda los valores calculados en caché.
class Factorial:

    def __init__(self):
        self.cache = {}

    def fact(self, number):
        if number == 0:
            return 1
        else:
            return number * self.fact(number-1)

    def __call__(self, number):
        if number in self.cache:
            return self.cache[number]
        else:
            result = self.fact(number)
            self.cache[number] = result
            return result

factorial = Factorial()
print(factorial(200))
print(factorial(2))
print(factorial(200))
print(factorial.cache)
\end{lstlisting}


\subsubsection{Código en Unity}

\begin{lstlisting}[style=Unity]
/* El contenido inicial de un archivo de código en Unity
   será algo así:*/
using UnityEngine;
using System.Collections;

public class MainPlayer : MonoBehaviour {
    // Use this for initialization
    void Start () {
    }
  
    // Update is called once per fríame
    void Update () {
    }
}
\end{lstlisting}


\subsubsection{Código Genérico}

\begin{lstlisting}
/* The code below will print the words Hello World
to the screen, and it is amazing */
Console.WriteLine("Hello World!"); 
/* def sum_list_limits_1(a, lower, upper): */
    if lower > upper:  % comment in Prolog
        return 0  # comment in Python 
    else:  // inline comment
        return a[upper] $+ \cup$ sum_list_limits_1(a, lower, upper - 1)
\end{lstlisting}


\subsection{Algoritmos}

\subsection{Figuras}

\subsection{Tablas} 


\subsubsection{Tablas largas y anchas}


\begin{landscape}
    \begin{longtable}{l  l  *{15}{c} }
      \caption[Análisis de la presencia de trastornos mentales en los Videojuegos]{Análisis de la presencia de trastornos mentales en los Videojuegos\\`-' no aparece; `a' aparece; `b' centrado.}
      \label{tab:larga}
      \endfirsthead
      \endhead  % header material
      \endfoot  % footer material

\toprule
                      & \multirow{2}{*}{}            & \multicolumn{4}{c}{Ansiedad} & \multirow{2}{*}{TCA} & \multirow{2}{*}{TAG} & \multirow{2}{*}{TPs} & \multirow{2}{*}{TLP} & \multirow{2}{*}{TEPT} \\ \cline{3-6}
                      &                              & TGA  & TAS  & Fobias & TOC   &                      &                   &                   &                   &                   \\
\midrule
                      & \citetitle{jekyll88}	     &      &      &        &       &                      &                   &                   &                   &                   \\
                      & Psychosis		     &      &      &        &       &                      &                   &                   &                   &                   \\
                      & Adventures of Smart Patrol   &      &      &        &       &                      &                   &                   &                   &                   \\
                      & Final Fantasy VII	     &      &      &        &       &                      &                   &                   &                   &                   \\
                      & Juggernaut		     &      &      &        &       &                      &                   &                   &                   &                   \\
                      & Sanitarium		     &      &      &        &       &                      &                   &                   &                   &                   \\
                      & Planet Laika		     &      &      &        &       &                      &                   &                   &                   &                   \\
                      & American McGee's Alice	     &      &      &        &       &                      &                   &                   &                   &                   \\
                      & Jekyll And Hyde		     &      &      &        &       &                      &                   &                   &                   &                   \\
                      & Silent Hill 2		     &      &      &        &       &                      &                   &                   &                   &                   \\
                      & Eternal Darkness	     &      &      &        &       &                      &                   &                   &                   &                   \\
                      & The Suffering		     &      &      &        &       &                      &                   &                   &                   &                   \\
                      & Killer7			     &      &      &        &       &                      &                   &                   &                   &                   \\
                      & Crush			     &      &      &        &       &                      &                   &                   &                   &                   \\
                      & Manhunt2		     &      &      &        &       &                      &                   &                   &                   &                   \\
                      & Haze			     &      &      &        &       &                      &                   &                   &                   &                   \\
                      & Batman: Arkham Asylum	     &      &      &        &       &                      &                   &                   &                   &                   \\
                      & Dreamkiller		     &      &      &        &       &                      &                   &                   &                   &                   \\
                      & Alice: Madness Returns	     &      &      &        &       &                      &                   &                   &                   &                   \\
                      & L.A. Noire	      	     &      &      &        &       &                      &                   &                   &                   &                   \\
                      & The Cat Lady	      	     &      &      &        &       &                      &                   &                   &                   &                   \\
                      & Far Cry 3	      	     &      &      &        &       &                      &                   &                   &                   &                   \\
                      & Actual Sunlight	             &      &      &        &       &                      &                   &                   &                   &                   \\
                      & Depression Quest	     &      &      &        &       &                      &                   &                   &                   &                   \\
                      & Outlast			     &      &      &        &       &                      &                   &                   &                   &                   \\
                      & Blackbay Asylum		     &      &      &        &       &                      &                   &                   &                   &                   \\
                      & Ether One		     &      &      &        &       &                      &                   &                   &                   &                   \\
                      & Evil Within		     &      &      &        &       &                      &                   &                   &                   &                   \\
                      & Neverending Nightmares	     &      &      &        &       &                      &                   &                   &                   &                   \\
                      & Batman: Arkham Knight	     &      &      &        &       &                      &                   &                   &                   &                   \\
                      & Darkest Dungeon		     &      &      &        &       &                      &                   &                   &                   &                   \\
                      & Fran Bow		     &      &      &        &       &                      &                   &                   &                   &                   \\
                      & Pry			     &      &      &        &       &                      &                   &                   &                   &                   \\
                      & Rise of the Tomb Raider	     &      &      &        &       &                      &                   &                   &                   &                   \\
                      & Sym			     &      &      &        &       &                      &                   &                   &                   &                   \\
                      & Until Dawn		     &      &      &        &       &                      &                   &                   &                   &                   \\
                      & Beginner's Guide             &      &      &        &       &                      &                   &                   &                   &                   \\
\midrule                                                  
\multirow{4}{*}{\rotatebox{90}{2010}} & Désiré       & -    &      &        &       &                      &                   &                   &                   &                   \\
                      & Layers of Fear               & -    &      &        &       &                      &                   &                   &                   &                   \\
                      & Tharsis                      &      &      &        &       &                      &                   &                   &                   &                   \\
                      & That Dragon, Cancer          &      &      &        &       &                      &                   &                   &                   &                   \\
\midrule                                                  
                      & The Town of Light            &      &      &        &       &                      &                   &                   &                   &                   \\
                      & Doki Doki Literature Club    &      &      &        &       &                      &                   &                   &                   &                   \\
                      & Fractured Minds              &      &      &        &       &                      &                   &                   &                   &                   \\
                      & Hellblade: Senua's Sacrifice &      &      &        &       &                      &                   &                   &                   &                   \\
                      & Last Day of June             &      &      &        &       &                      &                   &                   &                   &                   \\
                      & Night in the Woods           &      &      &        &       &                      &                   &                   &                   &                   \\
                      & Celeste                      &      &      &        &       &                      &                   &                   &                   &                   \\
                      & Gris                         &      &      &        &       &                      &                   &                   &                   &                   \\
                      & Heartbound                   &      &      &        &       &                      &                   &                   &                   &                   \\
                      & Life is Strange              &      &      &        &       &                      &                   &                   &                   &                   \\
                      & The Red Strings Club         &      &      &        &       &                      &                   &                   &                   &                   \\
                      & Sea of Solitude              &      &      &        &       &                      &                   &                   &                   &                   \\
                      & Tell Me Why                  &      &      &        &       &                      &                   &                   &                   &                   \\
                      & EL MÍO                       &      &      &        &       &                      &                   &                   &                   &                   \\
\bottomrule
                    \end{longtable}
\end{landscape}



\chapter{Referencias Bibliográficas}
\label{sec:bib}

Objetivos generales y específicos del trabajo.

\section{El archivo \emph{bibliografia.bib}}

En este capítulo se pueden añadir secciones, pero no son obligatorias
en capítulos cortos.

\section{El uso de cite y el estilo apalike}

El objetivo de este documento es proporcionar una plantilla de \LaTeX
para TFG. No debe usarse como sustituto de la normativa de TFG
aprobada por la ETSII.



\chapter{Inserción de imágenes}




\section{Primera sección}

% Citar una referencia
Esto es una referencia bibliográfica \citetitle{bibex}. Se recomienda
leer ``The Not So Short Introduction to \LaTeX'' \cite{Oetiker2007}
(existen versiones más modernas).


\subsection{Ecuaciones y fórmulas}

Gracias a la ecuación de Euler
($e^{ \pm i\theta } = \cos \theta \pm i\sin \theta$) podemos ver la
relación entre varias de las constantes matemáticas más importantes:
\[
    e^{i\pi} + 1 = 0.
\]


% Fórmula numerada
Si una ecuación se va a referenciar es necesario numerarla:
\begin{eqnarray}
\label{eq:schemeP}
 \Phi (k)=\dfrac{2}{|R(k)|(|R(k)|-1)} \underset{i,j \in R(k)}{\sum} a_{ij}.
\end{eqnarray}
Posteriormente se hace referencia a la ecuación a través de su
etiqueta (label). Por ejemplo, la anterior ecuación
\eqref{eq:schemeP}.


\subsection{Tablas y figuras}

% Insertar una tabla
\begin{table}
  \centering
  \caption{Título de la tabla.}
  \label{tab:una_tabla}

\begin{footnotesize}
%\renewcommand{\arraystretch}{1.5} % Para cambiar la separación entre filas (1 por defecto)
\begin{tabular}{ccccccccccc}
  \hline
   & Subs. & Students & A & PE & WA & RE & CTE & IF & TLE & All\\
  \hline
Ex. 1 & 104 & 44 & 1.27    &   0       &   0.55    &   0.23    &   0.20    &   0.11    &   0     & 2.36  \\
Ex. 2 & 118 & 37 & 0.92    &   0       &   0.92    &   0.27    &   0.49    &   0.59    &   0     & 3.19  \\
Ex. 3 & 100 & 28 & 1.21    &   0.39    &   1.18    &   0.54    &   0.14    &   0.07    &   0.04  & 3.57  \\
Ex. 4 & 78  & 25 & 1.08    &   0.84    &   0.52    &   0.40    &   0.24    &   0.04    &   0     & 3.12  \\
Ex. 5 & 116 & 31 & 1.48    &   0.10    &   0.77    &   0.32    &   0.42    &   0.19    &   0.45  & 3.74  \\
Ex. 6 & 213 & 32 & 1.06    &   0.34    &   3.81    &   0.56    &   0.69    &   0.06    &   0.13  & 6.66  \\
Ex. 7 & 116 & 34 & 1.35    &   0.38    &   0.38    &   0.68    &   0.62    &   0       &   0     & 3.41  \\
  \hline
Average & 120.7 & 33 & 1.20 &  0.26 &  1.14 &  0.42 &  0.40 &  0.16 &  0.08 & 3.66 \\
  \hline
 \end{tabular}
\end{footnotesize}

\end{table}









\begin{sidewaystable}
  \centering
  \caption{Tabla rotada. Factor groupings for the Mooshak questionnaire.}\label{tab:factor_analysis}

\renewcommand{\arraystretch}{1.1}
\begin{scriptsize}
 \begin{tabular}{clcc}
   \hline
   Factor & \textbf{Interpretation} / Items$^{*}$ (loadings)  & Median & Mode \\
   \hline
   \hline
    1 & \multicolumn{3}{l}{\textbf{Students' perception of Mooshak towards its helpfulness in learning} } \\
   \hline
    (21.17\%) & m10. Mooshak has forced me to implement programs more carefully $(0.849)$ & 4 & 4 \\
    $\alpha$ = 0.922 & m6.  Mooshak has helped me improve as a programmer $(0.819)$ & 3 & 4 \\
     & m5.  Mooshak has made me more aware of the need to write correct code $(0.781)$ & 3 & 3\\
     & m1. Mooshak has forced me to program more responsibly $(0.713)$ & 3 & 3 \\
     & m15. The specifications regarding the exercises used with Mooshak are adequate $(0.687)$ & 3 & 3 \\
     & m18. Mooshak helps to measure my current programming skills $(0.680)$ & 2.5 & 3 \\
   \hline
%   \multicolumn{4}{c}{} \vspace{-0.2cm}\\
%   \hline
    2 & \multicolumn{3}{l}{\textbf{Disposition towards using Mooshak} } \\
   \hline
    (17.93\%) & m24. I would be willing to participate in a programming contest using Mooshak, with similar exercises to the ones & 2 & 1 \\
    $\alpha$ = 0.897 & seen throughout the course $(0.807)$ & & \\
    & m13. Using Moohak in the final exams is a good idea $(0.748)$ & 2 & 1 \\
    & m14. I would like to use Mooshak or a similar tool in the future $(0.734)$ & 3 & 1 \\
    & m17. Knowing Mooshak can motivate me to take part in a programming contest $(0.655)$ & 2 & 1\\
    & m9. It would have been useful to use Mooshak from the first programming course $(0.527)$ & 2.5 & 1\\
     & m16. Using Mooshak in the course has been interesting $(0.522)$ & 3 & 4 \\
   \hline
%   \multicolumn{4}{c}{} \vspace{-0.2cm}\\
%   \hline
    3 & \multicolumn{3}{l}{\textbf{Effect of Mooshak's feedback in the tool's usefulness} } \\
   \hline
    (14.84\%) & m12. Mooshak's feedback is adequate $(0.832)$ & 2 & 1\\
    $\alpha$ = 0.836 & m3. Using Mooshak has increased my workload considerably $(0.693)$ & 4 & 4 \\
     & m7.  If Mooshak does not accept my code I feel motivated to find and fix the errors $(0.691)$ & 2 & 3 \\
     & m8.  In general, using Mooshak has been a good idea $(0.666)$ & 3 & 4 \\
   \hline
%   \multicolumn{4}{c}{} \vspace{-0.2cm}\\
%   \hline
    4 & \multicolumn{3}{l}{\textbf{Mooshak's effect on persistence} } \\
   \hline
    (11.20\%) & m23. When Mooshak does not accept my code I get discouraged and I abandon the exercise $(0.848)$ & 3 & 3 \\
    $\alpha$ = 0.705 & m22. Mooshak has been a waste of time $(0.597)$ & 2 & 2 \\
    & m25. Once a program has passed Mooshak's tests, I rewrite it in order to enhance it $(0.559)$ & 2 & 2 \\
   \hline
%   \multicolumn{4}{c}{} \vspace{-0.2cm}\\
%   \hline
   5 & \multicolumn{3}{l}{\textbf{Students' perception of Mooshak's features} } \\
   \hline
    (10.87\%) & m20. Even if it is not related to the grade, I feel satisfied if I am one of the first students to complete an exercise $(0.729)$ & 2 & 2\\
   $\alpha$ = 0.742  & m19. I value the fact that a tool like Mooshak returns feedback in real time about the correction of my programs $(0.650)$ & 3.5 & 4 \\
   \hline
%   \multicolumn{4}{c}{} \vspace{-0.2cm}\\
%   \hline
\multicolumn{4}{l}{\scriptsize $^{*}$Measured on a 5-point Likert scale (1: strongly disagree; 2: disagree; 3: neutral; 4: agree; 5: strongly agree).}
  \end{tabular}
\end{scriptsize}
\end{sidewaystable}



\begin{table}
  \centering

\begin{small}
\begin{tabular}{|l|l|l|l|}\hline
  \multirow{10}{*}{numeric literals} & \multirow{5}{*}{integers} & in decimal & \verb|8743| \\ \cline{3-4}
  & & \multirow{2}{*}{in octal} & \verb|0o7464| \\ \cline{4-4}
  & & & \verb|0O103| \\ \cline{3-4}
  & & \multirow{2}{*}{in hexadecimal} & \verb|0x5A0FF| \\ \cline{4-4}
  & & & \verb|0xE0F2| \\ \cline{2-4}
  & \multirow{5}{*}{fractionals} & \multirow{5}{*}{in decimal} & \verb|140.58| \\ \cline{4-4}
  & & & \verb|8.04e7| \\ \cline{4-4}
  & & & \verb|0.347E+12| \\ \cline{4-4}
  & & & \verb|5.47E-12| \\ \cline{4-4}
  & & & \verb|47e22| \\ \cline{1-4}
  \multicolumn{3}{|l|}{\multirow{3}{*}{char literals}} & \verb|'H'| \\ \cline{4-4}
  \multicolumn{3}{|l|}{} & \verb|'\n'| \\ \cline{4-4}          %% here
  \multicolumn{3}{|l|}{} & \verb|'\x65'| \\ \cline{1-4}        %% here
  \multicolumn{3}{|l|}{\multirow{2}{*}{string literals}} & \verb|"bom dia"| \\ \cline{4-4}
  \multicolumn{3}{|l|}{} & \verb|"ouro preto\nmg"| \\ \cline{1-4}          %% here
\end{tabular}
\end{small}

  \caption{Tabla con ``multicolumnas'' y ``multifilas''.}\label{tab:tablacompleja}
\end{table}





% Insertar una figura
\begin{figure}
  \centering
  \includegraphics[width=0.75\textwidth,clip=true]{urjc-logo}
  \caption{Logo de la Universidad.}
  \label{fig:logo_universidad}
\end{figure}

% Referenciar una etiqueta (label)
Las tablas y figuras deben presentarse en el texto, referenciadas y
numeradas. La descripción de una figura debe ir posicionada debajo de
la misma. Las descripciones de tablas pueden aparecer encima o debajo
de las mismas (pero de forma consistente en todo el documento).

En las tablas se recomienda evitar líneas verticales y usar pocas horizontales. 

La figura~\ref{fig:logo_universidad} se utiliza en la portada. \LaTeX
ubica automáticamente las tablas y figuras. Para ello emplea reglas
basadas en la experiencia de profesionales de la edición de
textos. Podemos forzar su ubicación, pero en general es recomendable
usar la ubicación sugerida por el sistema \LaTeX. Usad gráficos
vectoriales siempre que podáis.




\section{Segunda sección}

% Nueva página
Normalmente no tendremos que insertar saltos de página, salvo para forzar que los capítulos empiecen en páginas impares, con \begin{verbatim}\blankpage\end{verbatim} En cualquier caso, podemos introducir un salto de página con el comando \begin{verbatim}\newpage\end{verbatim}.

\newpage
% También con \pagebreak



\subsection{Código}

\begin{lstlisting}
/* The code below will print the words Hello World
to the screen, and it is amazing */
Console.WriteLine("Hello World!"); 
/* def sum_list_limits_1(a, lower, upper): */
    if lower > upper:  % comment in Prolog
        return 0  # comment in Python 
    else:  // inline comment
        return a[upper] $+ \cup$ sum_list_limits_1(a, lower, upper - 1)
\end{lstlisting}

% \begin{mypython}[float={!t},caption={Titulo del algoritmo/código.},label={alg:etiqueta}]
% def sum_list_limits_1(a, lower, upper):
%     if lower > upper:
%         return 0
%     else:
%         return a[upper] + sum_list_limits_1(a, lower, upper - 1)
% \end{mypython}
% El código~\ref{alg:etiqueta} es un ejemplo en Python.



\begin{algorithm}[t]
  \caption{\textit{Additional Louvain} \textbf{input}=$\left(A, \ \mathcal{M}\right)$ \textbf{output}=$P$}
  \begin{algorithmic}[1]
    \STATE $\forall i \in V$, \ let $i$ be an isolated community
    \STATE $o=permutation(V)$
    \FOR{$k \ \in \ o$}
    \STATE search in $A$ all the neighbours of $k$, $j$
    \STATE $\forall j$, calculate $\Delta Q_k(j)$ in matrix $\mathcal{M}$
    \STATE $j^*=\{ \ j \ | \ \Delta Q_k(j^*)=\max_j\{Q_k(j)\} \ \}$
    \IF{$\Delta Q_k(j^*)>0$}
    \STATE{Move node $k$ to $j^*$ 's community}
    \ELSE
    \STATE{$k$ remains in its community}
    \ENDIF
    \ENDFOR
  \end{algorithmic}
  \label{alg:AdditionalLouvain}
\end{algorithm}


En el algoritmo~\ref{alg:AdditionalLouvain} aparece un ejemplo en
pseudocódigo.  También se puede poner \code{x=1} inter-lineado.


\chapter{Resultados (opcional)}
\label{sec:resulObtenidos}


En esta sección se describe los resultados obtenidos en el TFG, en
caso de realizar propuestas para su resolución. Puede sustituirse por
ejemplos u omitirse.


\chapter{Conclusiones y trabajos futuros}

En este capítulo se detallan las conclusiones derivadas del TFG y la
propuesta de posibles trabajos futuros.

Las citas del texto Autor \cite{giaquinta}, Autor \cite{fortune},
Autor \cite{fortuneB}, Autor \cite{mitchell} y Autor \cite{morrey}
deben ir referenciadas en la bibliografia.



%%%%%%%%%%%%%%%%%%%%%%%%%%%%%%% Bibliografía 
\cleardoublepage
\phantomsection
\addcontentsline{toc}{chapter}{Bibliografía}

\printbibliography


%%%%%%%%%%%%%%%%%%%%%%%%%%%%%%% Apéndices
\cleardoublepage
\appendix

\chapter{Apéndice}


\section{Código del archivo \emph{bibliografia.bib}}
\label{app:bib}

\lstinputlisting{bibliografia.bib}




\end{document}
